\documentclass[executivepaper,10pt]{book}

\usepackage[latin1]{inputenc}
\usepackage[english]{babel}
\usepackage{fontenc}
\usepackage{graphicx}

\title{OpenDAX User Manual}
\date{June 13, 2008}
\author{Phil Birkelbach}


\begin{document}
\begin{titlepage}
\maketitle

\begin{flushleft}
Copyright \textcopyright 2008 - Phil Birkelbach\linebreak
All Rights Reserved
\end{flushleft}

\end{titlepage}
\chapter*{Introduction}
OpenDAX is an open source, modular, data acquisition and control system. It is licensed under the GPL (GNU Public License) and therefore is completely free to use and modify. OpenDAX is written primarily for Free operating systems like Linux, BSD and Darwin (Mac OS X). We are making every effort to make the code as portable as possible but these are the big three that we will do development on.

OpenDAX could be used for anything from controlling the air conditioner in a home to controlling an entire industrial facility. Depending on what modules are installed and run it could be used as the firmware for a dedicated Programable Logic Controller (PLC) or a Home Automation system. It could loosely be compared to DCS (Distributed Control System) or a SCADA (Supervisory Control and Data Acquisition) system. Eventually the system would be able to scale up to a several hundred thousand tag system. At this moment the code is far to immature to be used for anything that required reliability but we hope to get it to that point sooner or later. Much will depend on how many developers take up the challenge to help work on the code.

DAX stands for Data Acquisition and eXchange. The system works by combining a backend program (opendax) a library (libdax) and set of modules. The opendax server handles the managing of the database, the messaging and the modules.  The database is an area of memory that holds the real time data that the system is using.  It could hold the temperature or pressure of a process, the status of a switch or some command data from the Human Interface to a logic module.  Messaging (in this context) is the low level communication between modules and the server or between modules.  Actually there isn't any intermodule communication\footnote{There is nothing stopping two modules from communicating to each other with their own protocol} in the system but rather all communication is between the modules and the server and the server can act as a proxy to forward those messages to another module.

The server is also reponsible for starting, stopping and managing the modules.  The modules are simply process that run as children to the server.\footnote{It is possible for other processes to start modules but the server will have much less control over them}  The server can automatically start these modules in the proper order and can also restart them if they crash.

The modules do all the work and communicate with the opendax server through the libdax library. There will be modules for reading and writing to I/O points, data logging, alarming, machine interface, and logic. It will even be possible to use programs that were never written to be OpenDAX modules. These programs can be started from the OpenDAX process and have their STDIN and STDOUT piped to other modules so that they can be controlled. This will keep the OpenDAX system from having to reinvent the wheel if there is an existing program to do the work. Some examples might be using 'dc' for abitrary precision math, or mpg123 as a tone generator. Any program that you can interact with from the command line should work with OpenDAX as a module. 

\chapter*{Getting Started}
First, download OpenDAX. See the download page for details on how to get the source code.

If you get the source code from the Subversion repository you will probably have to run the bootstrap.sh file in the root directory of the package to set up autoconf and automake. Otherwise you should be able to simply run...

\begin{verbatim}
    ./configure
    make
    sudo make install 
\end{verbatim} 
This should install the program on most operating systems but since OpenDAX is still in the early development stage it is likely that there will be problems. Please help us figure out how to get autoconf and automake to work properly on the system that you are installing on.

OpenDAX has two library dependencies at this point. The readline library is used by the daxc command line module. If it is not there then daxc should still compile but with reduced functionality.

The second dependency is Lua Version 5.1. Lua libraries will likely be a problem. All the modules and the opendax server itself use Lua as the configuration file language.

One of the problems with the lua libraries is that different distributions will install the libraries with different names for the libraries and header files. The configure script tries to figure out where they are but you might have to help to get configure to find them.

Configure will look for libraries in the ld search path with these names, liblua, liblua51 and liblua5.1. If your distribution has another name for the libraries please let us know.

Configure will look for the header files, lua.h, luaxlib.h and lualib.h in the directories lua/ lua51/, lua5.1/ and in the normal include directories. If it doesn't find any of these there will likely be compile time errors, when building lua.
OS Specific Notes
\section*{Mac OS X}
Download the source code file from www.lua.org and uncompress the file somewhere on your hard disk. At the time of this writing the lastet version was 5.1.2.
\begin{verbatim}
    % tar -xzvf lua-5.1.2.tar.gz
\end{verbatim}

Then it's a simple matter of...
\begin{verbatim}
    cd lua-5.1.2
    make macosx
    sudo make install
\end{verbatim}

This is the easiest way that i have found to satisfy the Lua library requirements on OS X. This statically links the library but it's tiny so that shouldn't be too much of an issue. This is good enough for development at this point.

The readline library should already be installed with OS X and shouldn't cause a problem.
\section*{Ubuntu Linux}

Install the following packages...
\begin{verbatim}
    % sudo apt-get install liblua5.1-0-dev lua5.1 lua5.1-doc
\end{verbatim}
Lua depends on libreadline as well so once these packages are installed all of the OpenDAX dependencies should be met. This was last tried on Ubuntu 7.10.
\section*{FreeBSD}

I've had trouble getting the Lua dependencies met with FreeBSD. I can get the libraries installed and configure finds them but the linker doesn't find them when make is run for some reason. I don't know that much about FreeBSD and frankly I don't use it so I had little motivation to work out the problems. I'd love it if someone could figure it out and drop me a line so that I can include those notes here. 
\end{document}
