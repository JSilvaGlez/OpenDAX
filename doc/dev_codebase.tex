\chapter{Codebase Descriptions}
This section describes the different files and directories that are part of the OpenDAX core program (the server and the library).  This is meant as a rough overview to get a feel for how the code is laid out.  It is not meant to be an exhaustive reference.  For that you'll just have to go take a look at the latest code in the repository.  As a module developer you may not need any of this information, but since OpenDAX is such a young program there will still be plenty of bugs to work out of the core system, and this list may help somebody to get into the core code and help out with debugging or development of the server or library.

\emph{/config.h}
This file is generated by ./configure and contains the architecture dependent defines that tell us things like what functions are available and what header files we need to install. I'm actually checking for quite a few functions that I don't really do anthing about if they are missing. As incompatibilities come up these will have to be dealt with.

\emph{/common.h}

This header includes config.h as well as some other headers that are popular. It also has some definitions and macros that will be used throughout the system. This file should be included in just about every source code file in the system.

\emph{/opendax.h}

This is the header that describes the public interface to the OpenDAX library. It contains declarations for all of the public library functions as well as the precompiler definitions for the data types, error codes, configuration flags etc. This file should be included by all modules that will link to the library, and is included in most of the source code files in the rest of the system too. This should be the only header file from this distribution that would need to be included in any module code. If there are others then we did something wrong in the interface.

\emph{/dax}

The dax directory contains the source for the OpenDAX server.

\emph{/dax/daxtypes.h}

This contains the private type definitions that are used internally by the server. These definitions should not be used by any module or the library.

\emph{/dax/libcommon.h}

This file contains type definitions that are common between the library and the server. It should be included in the server and the libary source.

\emph{/dax/func.c}

Contains a few generic functions for common operations like memory allocation and such.

\emph{/dax/module.c}

Contains the code for the module handling system in the server. Operations such as starting and stopping modules as well as module registration and any other operation that involves modules should be in this file. The module.h header contains the public interface definitions for the functions in this file.

\emph{/dax/message.c}

This file contains the functions that handle the module<->server messaging. Very little actual work gets done here other than sending and receiving data on the sockets and determining which functions in other files should be called.

\emph{/dax/buffer.c}

Contains buffering code for the messaging subsystem. The messaging system writes the data from each socket into these buffers until it figures out that it has an entire message from one of the modules then it calls the function to deal with that message.

\emph{/dax/options.c}

Contains the code for reading the configuraiton from the configuration file and the command line.

\emph{/dax/opendax.c}

Contains main() and the other functions necessary to start the server and spawn the threads that do all the work. It all starts here.

\emph{/dax/tagbase.c}

The storage and maipulation of all the real time tag information is contained within this file. For now the custom datatype handling code is also in here but this may get moved to it's own file.

\emph{/lib}

The lib directory contains the source for the library. We use libtool to sort out the compatibility issues associated with the way different systems handle shared libraries. If no shared libary system will work then libtool makes this a static library.

\emph{/lib/libdax.h}

This header file contains all of the private definitions, macros and declarations that are needed throughout the library code but are not needed in the server or the modules.

\emph{/lib/libfunc.c}

Contains some generic functions that are useful throughout.

\emph{/lib/libmsg.c}

This is all the messaging code for the library. There are functions in this file for sending and retrieving messages and determining what to do with them. The functions here closely resemble functions in /dax/message.c and typically changes in one of thse files requires changes in the other. Between the two of them they define the
communications protocol.

\emph{/lib/data.c}

This file deals with library side of the tag data. It is probably not named well and may change. The functions here resemble functions that are in /dax/tagname.c

\emph{/lib/libconv.c}

This file contains the functions for making sure that the data formatting is the same as the server. The way that OpenDAX handles different byte ordering and data formating between architectures over the network is that the server stores the data in whatever way the server wants and the library is responsible for determining if the data needs to be converted and how. This file contains the code for that.

\emph{/lib/libcdt.c}

OpenDAX allows the creation of custom data types. This file contains the code to handle all of that.

\emph{/lib/libopt.c}

Contains the functions for configuring the module. The entire configuration system of OpenDAX uses Lua as the configuration programming language. The modules can be configured by either the main opendax.conf file, their own configuration file, the command line or any combination of the above. There are also some configuration options that are common between modules. This file contains the code to handle all of that.

\emph{/modules}

This directory contains the module code. We will not discuss the module details here.

\emph{/etc}

The sample configuration files are located here.
