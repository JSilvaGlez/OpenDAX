\documentclass[10pt,letterpaper]{report}

\usepackage[american]{babel}
\usepackage{amsmath}
\usepackage{amsfonts}
\usepackage{amssymb}
\usepackage{setspace}
\usepackage{parskip}
\usepackage[dvips]{graphicx}
\usepackage{float}
\usepackage{makeidx}
\usepackage{verbatim}


\setlength\parskip{10pt}
\setcounter{secnumdepth}{3}

\makeatletter
\def\thickhrulefill{\leavevmode \leaders \hrule height 1pt\hfill \kern \z@}
\renewcommand{\maketitle}{\begin{titlepage}%
    \let\footnotesize\small
    \let\footnoterule\relax
    \parindent \z@
    \reset@font
    \null\vfil
    \begin{flushleft}
      \huge \@title
    \end{flushleft}
    \par
    \hrule height 1pt
    \par
    \begin{flushright}
      \LARGE \@author \par
    \end{flushright}
    \vskip 60\p@
    \vfil\null
  \end{titlepage}%
  \setcounter{footnote}{0}%
}

\makeatother
\makeindex

\def\opendax{\textit{OpenDAX}}
\def\modbus{\textit{Modbus}$\textsuperscript{\textregistered}$}
\def\daxstate{\texttt{dax\_state} }
\def\eventadd{\texttt{dax\_event\_add()}}
\def\eventdel{\texttt{dax\_event\_del()}}
\def\eventselect{\texttt{dax\_event\_select()}}
\def\eventpoll{\texttt{dax\_event\_poll()}}
\def\eventgetfd{\texttt{dax\_event\_get\_fd()}}
\def\eventdispatch{\texttt{dax\_event\_dispatch()}}


\title{OpenDAX Developer's Guide}
\date{July 15, 2008}
\author{Phil Birkelbach}

\begin{document}
\pagenumbering{roman}
\maketitle
\begin{flushleft}
Copyright \textcopyright 2008 - Phil Birkelbach\linebreak
All Rights Reserved

\end{flushleft}

\tableofcontents
\newpage
\pagenumbering{arabic}

\chapter{Introduction}
\opendax is an open source, modular, data acquisition and control system. It is licensed under the GPL (GNU General Public License) and therefore is completely free to use and modify.

This book is a tutorial for developing modules for OpenDAX as well as a reference for experienced module developers.  It should be noted that, as of this writing, OpenDAX is very immature and much of the information in this book may have changed.  The developers attempt to keep the interface constant, but at this stage of development it will make sense from time to time to make changes that would break existing module code as well as conflict with information in this book.  The current source code is the ultimate authority on the API.


\section{OpenDAX Basics}
Before we get started, it would be good to discuss how OpenDAX works.  OpenDAX is made up of three main parts, the server, the library and the modules.

The OpenDAX server\index{OpenDAX server} is the heart and soul of the OpenDAX system.  It sits at the center of the OpenDAX universe and coordinates all of the data and communications of the system.  The libdax library\index{libdax library} abstracts the communication interface to the modules and the modules are where all the work is done.

If the server is the heart and soul of the OpenDAX application then modules are the arms and legs.  Modules do all of the work.  They are separate processes that may or may not be started by the server at run time.  Modern operating systems do a great job of managing processes and the OpenDAX developers did not see any need to reinvent that wheel.  

Modules handle all of interface to the outside world.  The application logic, any storage or logging functions and the human interface are all handled by modules.  The modules all communicate to the OpenDAX server through an API that is exposed in the libdax library.  The libdax library API is what the OpenDAX module developer will see of OpenDAX.

The low level communications to the OpenDAX server takes place through a BSD Socket interface.  There is no requirement that the module be running on the same machine as the server.  The entire application can be distributed however the application developer desires.  If the module is on the same machine, it can communicate to the server via UNIX Domain Socket.  If the module is on a remote host then it will have to use a TCP Socket.  The UNIX Domain sockets are considerably faster than TCP since they are really nothing more than a memory copy within the kernel.

The exact nature of the communication is subject to change at this point so I won't go into that in too much detail here.  Right now there are two sockets that are created for each module.  The primary socket allows the module to send and receive data to the server, the second socket is used for asynchronous event reporting.

At some point in the future we hope to implement some server-to-server communication that might facilitate redundancy or an even better ability to distribute the system.  Right now there is only one server in any application and all the modules communicate to that single server.

Modules can be started by the OpenDAX server or by any other means that the operating system has for starting processes (i.e. shell prompt, scripts etc).  If the module is to be started by the server there are a few advantages.  First the server will know if the module dies for some reason because the operating system will send it a signal.  This allows the server to restart the module or alert the user that a part of the system is down.  Another advantage of having the module be a child process of the server is that the server would have access to the stdin, stdout and stderr file descriptors and could redirect the I/O from the module in interesting ways.  This feature allows programs that were never meant to be OpenDAX modules to be used with and controlled by OpenDAX\footnote{This feature is not fully implemented yet but it should be very soon}.

The server contains the tag database.  A tag\index{tag} is the atomic unit of data in the system.  These tags are analogous to variables in a programming language.  There are many different data types in OpenDAX and the user or module programmer can create compound data types that are collections of other types.  Compound data types are similar to a structure in C.  We will discuss compound data types later in this text.

The tag database contains the names of these tags, their data type\index{data type}, the actual real time value and the events that the tag responds too.  It is the central store house of information in OpenDAX.  Different modules work with tags in different ways.  For instance, the \modbus \index{Modbus}module reads data from one or more \modbus devices and stores that information in tags within the server.  The tags are arrays of type UINT or BOOL\footnote{UINT is an unsigned 16 bit data type, BOOL is a single bit data type} depending on the command.  The names of these tags are a configurable parameter of the \modbus module.  If and HMI or Logic module need the information from the \modbus module they would read these tags to get it.

The server also contains information about the currently running modules.  Modules must register with the server when they are started before the server will answer any queries by the module.  This registration step is how the module and the server learn what they need to learn about each other to properly communicate.  Once registered the server can keep track of that module through the file descriptor of the socket on which the the connection is made.  This information can be made available to other modules that may need it.

\section{Installation}

First, download OpenDAX. See the download page on the www.opendax.org website for details on how to get the source code.  Right now the only distribution format is a Gzipped Tar file.  You can unzip this file with the following command \ldots

\verb|tar -xzvf opendax-0.4|

This will unarchive the file and put it in the ./opendax-0.4 directory.  The version number may be different.  0.4 was the version as of this writing.

You can also download the program from the Subversion repository with this command \ldots

\begin{verbatim}
svn co https://opendax.svn.sourceforge.net/svnroot/opendax opendax
\end{verbatim}
The subversion username is \texttt{guest} and the password is blank.

If you get the source code from the Subversion repository you will have to run the bootstrap.sh file in the root directory of the package to set up autoconf and automake. Otherwise you should be able to simply run...

\begin{verbatim}
  cd opendax-0.4
  ./configure
  make
  sudo make install 
\end{verbatim} 

This should install the program on most operating systems but since \opendax is still in the early development stage it is likely that there will be problems. Please help us figure out how to get autoconf and automake to work properly on the type of system that you are using.

OpenDAX has two library dependencies at this point. The readline library is used by the daxc command line module. If not there, daxc should still compile but with reduced functionality.

The second dependency is Lua Version 5.1. Lua libraries will likely be a problem. All the modules and the opendax server itself use Lua as the configuration file language.

One of the problems with the lua libraries is that different distributions will install the libraries with different names for the libraries and header files. The configure script tries to figure out where they are but you might have to help to get configure to find them.

Configure will look for libraries in the ld search path with these names, liblua, liblua51 and liblua5.1. If your distribution has another name for the libraries please let us know.

Configure will look for the header files, lua.h, luaxlib.h and lualib.h in the directories lua/ lua51/, lua5.1/ and in the normal include directories. If it doesn't find any of these there will likely be compile time errors.

\subsection{Mac OS X}
Download the source code file from www.lua.org and uncompress the file somewhere on your hard disk. At the time of this writing the lastet version was 5.1.4.
\begin{verbatim}
  tar -xzvf lua-5.1.4.tar.gz
\end{verbatim}

Then it's a simple matter of...
\begin{verbatim}
  cd lua-5.1.4
  make macosx
  sudo make install
\end{verbatim}

This is the easiest way that I have found to satisfy the Lua library requirements on OS X. This statically links the library but it's tiny so that should not be too much of an issue. This is good enough for development at this point.

The readline library should already be installed with OS X and should not cause a problem.

If you downloaded the program from the Subversion repository you may have to modify the \emph{bootstrap.sh} file.  OS X renamed the \emph{libtoolize} program for some reason.  In OSX this program is named \emph{glibtoolize}.  Make sure this line in \emph{bootstrap.sh} is correct or you will get errors.  If you downloaded the distribution file you do not have to worry about the \emph{bootstrap.sh} file at all.
\subsection{Ubuntu Linux}

Install the following packages...
\begin{verbatim}
  sudo apt-get install liblua5.1-0-dev lua5.1 lua5.1-doc
\end{verbatim}
Lua depends on libreadline as well so once these packages are installed all of the OpenDAX dependencies should be met. This was last tried on Ubuntu 9.10.
\subsection{FreeBSD}

I have had trouble getting the Lua dependencies met with FreeBSD. I can get the libraries installed and configure finds them but the linker does not find them when make is run for some reason. I don't know that much about FreeBSD and frankly I don't use it so I had little motivation to work out the problems. I'd love it if someone could figure it out and drop me a line so that I can include those notes here. 


\section{Setting up the Build Environment}
The build environment for developing modules is pretty simple.  If you have installed OpenDAX then you should have everything that you need to compile and run OpenDAX modules.

There is no separate development package for \opendax.  All the files that you need to develop modules should be installed when you install \opendax from the distribution.

The library should be installed in a typical place (usually /usr/local/lib) and the \textit{opendax.h}\index{opendax.h} header file should be in a place where your compiler can find it (usually /usr/local/include).  For writing a module in C this should be all that you need.  If you have problems with the installation, see the \emph{OpenDAX User's Manual}, it has much more detailed information on what is needed to get OpenDAX up and running on your system.

You should be able to use any C compiler to build modules for OpenDAX, but we have been using GCC for the main development.

For all intents and purposes your modules sole interaction with the \opendax system is through the libdax library.  You should include the \emph{opendax.h} header file in your module's source code file and you should link the libdax library with your module with the \verb|-ldax| option to the compiler.\footnote{This is for \texttt{gcc}. Other compilers may have different options for linking shared libraries.}

At some point we intend to include Lua and Python bindings for the libdax library so that modules can be written in those two languages.  The Lua bindings will come first because that language is already integrated into most of OpenDAX's workings, so most of that code is already written.  It just needs to be organized and separated into it's own library.  Python should prove to be a fairly simple translation of the Lua code once that day comes.

\chapter{Configuration and Initialization}
The first thing that the module code should do is read it's configuration.  It is important to note that the module will have to read some configuration even if the module itself does not have any configurable options.  This is because the libdax library will need some configuration information to help it know how to communicate with the server.

There are three methods for the module to receive configuration information.  The main \textit{opendax.conf} contains configuration information for the server as well as any configuration that is common to all the modules on the local host machine.  Each module may have it's own configuration file that can be read and each module can receive configuration information passed as command line arguments.

There is a precedence between these configurations.  If a configuration attribute is present on the command line it will take precedence over that same attribute given in either of the configuration files.  Likewise if an attribute exists in both of the configuration files then the module specific file will take priority.  This gives us a way to set up global default parameters for all modules such as IP address and communication port of the server in the \textit{opendax.conf} file.  Then if any of the global data needs to be different for any given module, that information can be changed in the module specific file.  Further, if the information needs to be different for any given invocation of the module that can be passed on the command line.

Configuration is made up of attributes.  An example of an attribute would be the name of the local unix domain socket that the module should use to communicate to the server.  The attribute name might be \texttt{socketname} and the value \texttt{/tmp/opendax}\footnote{In fact these are the defaults for the local socket configuration}.  There are many attributes that are built into the system that will always need to be configured.  These are necessary for the basic functionality.  The \texttt{socketname} attribute is just one example.  The module developer can add as many attributes as are needed for configuring the functionality of the given module.  The libdax library contains interfaces to help the module developer with all of this.

If the configuration data is more complex than simply assigning a value to an attribute the is a callback function mechanism built into the configuration system.  The configuration files in OpenDAX are nothing more than Lua scripts.  The libdax library gives the module programmer the ability to create functions that can be called from the script.  These functions are programmed in the module code just like any other Lua extension function.  The full description of how to write Lua extension functions is beyond the scope of this book, but we will give some simple examples that would help with basic configuration.

\section{Initialization}

The first thing that any module will have to do is allocate and initialize a \verb|dax_state|\index{dax\_state object} object.  This is simply a matter of calling the \verb|dax_init()|\index{dax\_init() function} function.  For example see the following code.

\begin{verbatim}
dax_state *ds;

ds = dax_init("MyModule");
if(ds == NULL) exit(-1);
\end{verbatim}

First we declare a pointer to a \verb|dax_state| object.  This is an opaque pointer that you program will need to maintain for it's entire lifetime.

Once the pointer is declared we use the \verb|dax_init()| function to allocate it and initialize it.  The only real error that this function can really have is memory allocation failure, in which case it will return NULL.  Your module should probably log an error an exit gracefully.  The string that is passed to \verb|dax_init()| is the name of your module as it will be seen throughout the system.  We'll talk about this later.

Once we have initialized the \verb|dax_state| we need to allocate and initialize the configuration information in the \verb|dax_state|.  We do this with the \verb|dax_init_config()| \index{dax\_init\_config() function}function.  We'll add the following lines to our module code\ldots

\begin{verbatim}
int result;

result = dax_init_config(ds, "MyModule");
if(result) exit(result);
\end{verbatim}

The first argument to \verb|dax_init_config()| is the pointer to the \verb|dax_state| that we allocated earlier.  We will be seeing this a lot.  Just about every function in the library takes this pointer as the first argument.

The second argument is the module name.  This should be the same as was passed to \verb|dax_init()|.  It is used differently in \verb|dax_init_config()|.  Here it will be used to determine the filename of the modules configuration file\footnote{This is probably not the best way to do this so we reserve the right to change it in the future}.  For most modules this should probably be the same as the module name.

\section{Creating Attributes}
Attributes\index{Attributes} are the names that are given the different options that you can create in an OpenDAX configuration.  For instance, your \textit{fooserial} module will need to know what serial port to use, what baud rate to use and the communications settings it will need.  These would be attributes that you would like for the libdax library configuration system to find for you.

To add and attribute to the configuration state we use the \verb|dax_add_attribute()|\index{dax\_add\_attribute() function} function.  Here is the prototype.

\begin{verbatim}
int dax_add_attribute(dax_state *ds, char *name, char *longopt,
                      char shortopt, int flags, char *defvalue);
\end{verbatim}

The \textit{name} argument is the name that will be used for the attribute.  It would be the left side of the attribute assignment in the configuration file.  For instance if we use \texttt{"baudrate"} for our attribute name the configuration file might look like\ldots

\begin{verbatim}
baudrate = 9600
\end{verbatim}

The \textit{longopt} argument is the long name of command line argument that would represent this attribute.  If we use \texttt{"baud-rate"} for \textit{longopt} then we might set the baud rate by the following command\ldots

\begin{verbatim}
$ fooserial --baud-rate=9600
\end{verbatim}
%$
The \textit{shortopt} argument is a single character that works just like \textit{longopt}\footnote{For more information on how these two types of command line options work, refer to the documentation for the \textit{getopt} library.}

The \textit{defvalue} argument is the default value that will be assigned to this attribute if it is not set by any of the three configuration mechanisms.

The \textit{flags} argument is a bitwise OR {|} of the following definitions \ldots

\begin{tabular}{|l|l|}
\hline \verb|CFG_ARG_NONE| & No Arguments \\ 
\hline \verb|CFG_ARG_REQUIRED| & Argument is required \\ 
\hline \verb|CFG_ARG_OPTIONAL| & Argument is optional \\ 
\hline \verb|CFG_CMDLINE| & Parse Command line \\ 
\hline \verb|CFG_DAXCONF| & Parse opendax.conf file \\ 
\hline \verb|CFG_MODCONF| & Parse [module].conf file \\ 
\hline \verb|CFG_NO_VALUE| & Don't store a value, only call callback \\ 
\hline 
\end{tabular} 

The first three flags correspond to the \textit{getopt} library's usage of command line arguments.  If there should be no arguments to the attribute then \verb|CFG_ARG_NONE| should be used.  This would come in handy if the argument is simply a flag of some kind.  The -V option to many programs simply cause the program to print it's Version number and exit, is one example of an option that would take no arguments.

If an argument is required the flag \verb|CFG_ARG_REQUIRED| should be used.  The baud rate option in the above example would be pretty meaningless without a number of some kind to use as the baud rate.

The \verb|CFG_ARG_OPTIONAL| flag is used for an attribute that doesn't need an argument but where one might make sense.  A verbosity option when used alone might simply increase the verbosity of the programs output slightly but passing a numerical argument would increase the verbosity by that amount.

The next three flags, \verb|CFG_CMDLINE|, \verb|CFG_DAXCONF| and \verb|CFG_MODCONF| have to do with which of the three configuration sources this particular attribute could be found.  It may make sense to search for an attribute on the command line, in the \textit{opendax.conf} file and in our \textit{fooserial.conf} configuration file.  In this case we would put all three values in our \textit{flags} argument.  It may however only make sense to see command line options.  In this case we would simply use \verb|CFG_CMDLINE|. 

The final flag,\verb|CFG_NO_VALUE| can also be OR'd with the others.  This is used in the special case where we are not actually interested in any value that might be passed to the attribute in a configuration file.  The configuration system has the ability to call a callback function when attributes are set in configuration files.  If this callback function is the only thing that we are interested in, we can use this flag to save a little memory.

There are some attributes that are predefined for use by the libdax library.  You cannot use any of the names, long options or short options of these attributes.  These are listed in the following table.

\begin{tabular}{|l|l|c|}
\hline \textbf{name} & \textbf{longopt} & \textbf{shortopt} \\
\hline socketname & socketname & \texttt{S} \\
\hline serverip & serverip & \texttt{I} \\
\hline serverport & serverport & \texttt{P} \\
\hline server & server & \texttt{s} \\
\hline debugtopic & topic & \texttt{T} \\
\hline name & name & \texttt{N} \\
\hline cachesize & cachesize & \texttt{z} \\
\hline msgtimeout & msgtimeout & \texttt{o} \\
\hline config\footnotemark & config & \texttt{C} \\
\hline confdir\footnotemark[\value{footnote}] & confdir & \texttt{c} \\
\hline 
\end{tabular} 
\footnotetext{These are command line only attributes}

If your module tries to use any of these names or options the \verb|dax_add_attribute()| function will return an error.  This list is also subject to change.  If you want to know the absolute latest version of this list see the \textit{/lib/libopt.c} source code file in the \opendax distribution.

\section{Creating Callbacks}
\begin{verbatim}
int dax_attr_callback(dax_state *ds, char *name,
                      int (*attr_callback)(char *name, char *value));
\end{verbatim}

The \verb|dax_attr_callback()|\index{dax\_attr\_callback() function} function is used to add a callback function to the configuration system that will be called when this attribute is set.

[[Still working on this]]

\section{Writing Lua Functions}

\begin{verbatim}
int dax_set_luafunction(dax_state *ds, int (*f)(void *L), char *name);
\end{verbatim}

The \verb|dax_set_luafunction()|\index{dax\_set\_luafunction() function} function is used to set a function that can be called from your module configuration script.  This gives your module a lot of power in how it is configured.  A full explanation of writing Lua functions is beyond the scope of this book.  Review the Lua documentation for more information.

[[Still working on this]] 

\section{Running the Configuration}
To execute the configuration use the following function \ldots
\begin{verbatim}
int dax_configure(dax_state *ds, int argc, char **argv, int flags);
\end{verbatim}

The \verb|dax_configure()|\index{dax\_configure() function} function will run the configuration.  You pass this function the \textit{argc} and \textit{argv} variables that were passed to your module in \verb|main()|.

The \textit{flags} argument is a bitwise OR of \verb|CFG_CMDLINE|, \verb|CFG_DAXCONF| or \verb|CFG_MODCONF|.  These do just what you would think they would do.  Depending on which of these flags that you set the corresponding configuration mechanism will be used.  To cause the module to only read from the command line you would simply set the \verb|CFG_CMDLINE| flag. If you want either of the two configuration files to be run you would set one or both of the others.

\section{Retrieving Attributes}

Once we have run the configuration we use the \verb|dax_get_attr()|\index{dax\_get\_attr() function} function to retrieve the values that were set.

\begin{verbatim}
char *dax_get_attr(dax_state *ds, char *name);
\end{verbatim}

This is a very simple function that takes the \textit{name} of the attribute that you want and returns a pointer to the string.  This string is allocated within the \verb|dax_state| object and should not be modified.  If your module needs to store this string for later use you should make a copy of it\footnote{The \texttt{strdup()} function works well for this}.  The pointer will point to invalid information after the configuration has been freed.  We'll discuss freeing the configuration shortly.

\section{Setting Attributes} 

Setting attributes may seem silly at first glance.  After all, if we already know what the configuration options should be why would we call the system to start with.  Well, remember back at the beginning of the chapter where we discussed the fact that the library needs some information to do it's job.  You can use the \verb|dax_set_attr()|\index{dax\_set\_attr() function} function to set those outside of the configuration system.

\begin{verbatim}
int dax_set_attr(dax_state *ds, char *name, char *value);
\end{verbatim}

The prototype should be pretty self explanatory.  The \textit{name} argument should point to the name of the attribute you want to set and \textit{value} should point to the value that you want the attribute to take.

It is important to note that any callbacks that are associated with this attribute will be called as well.  This may have some usefulness.  Most modules will not need to set attribute values.

\section{Finishing Up}

Once we are done with the configuration we can use the \verb|dax_free_config()|\index{dax\_free\_config() function} function to free the configuration memory.

\begin{verbatim}
int dax_free_config(dax_state *ds)
\end{verbatim}

This function simply takes a pointer to the \verb|dax_state| object and frees the data associated with the configuration.  There are a lot of strings that are maintained by the configuration system in the library and this is nothing more than a way to free up that memory.  If your module will need access to these configuration options and you don't want to make copies then you do not need to call this function.

After calling this function any strings that you received from \verb|dax_get_attr()| will no longer be valid, so you don't want to reference those pointers any longer.

\section{Module Registration}
Before the server will communicate to the module, the module will need to register with the server.

There are two functions that deal with module registration.
\begin{verbatim}
int dax_mod_register(dax_state *ds, char *name)
int dax_mod_unregister(dax_state *ds)
\end{verbatim}

The \verb|dax_mod_register()| function\index{dax\_mod\_register() function} makes the initial connection to the server, identifies the module to the server and takes care of any other initialization issues that need to be handled.  Once the module has successfully been registered it can begin doing it's job.

The \emph{name} argument is the name that the server will use to identify this module.  Module names must be unique.  \verb|dax_mod_register()| returns 0 on success and an error code on failure.

The \verb|dax_mod_unregister()| function\index{dax\_mod\_unregister() function} informs the server that we are through and closes the connection.  It is a bad idea to let a module die without first calling \verb|dax_mod_unregister()|.


\chapter{Module Registration}

\chapter{Dealing with Data}
The basic unit of data in OpenDAX is the \textit{Tag}.  A tag is similar to a variable in a programming language.
\section{Data Types}
Tags can be one of 15 different data types in OpenDax.  These are given in the following table.

\begin{tabular}{|l|l|l|l|l|}
\hline \textbf{Name} & \textbf{Description} & \textbf{Size (bits)} & \textbf{Min} & \textbf{Max} \\
\hline BOOL & Boolean (True/False) & 1 & 0 & 1 \\
\hline BYTE & Bit String & 8 & 0 & 255 \\
\hline SINT & Signed Short Integer & 8 & -128 & 127 \\
\hline WORD & Bit String & 16 & 0 & 65,535 \\
\hline INT & Signed Integer & 16 & -32,768 & 32,767 \\
\hline UINT & Unsigned Integer & 16 & 0 & 65,535 \\
\hline DWORD & Bit String & 32 & 0 & 4,294,967,296 \\
\hline DINT & Double Integer & 32 & -2,147,483,648 & 2,147,483,647 \\
\hline UDINT & Unsigned Double Integer & 32 & 0 & 4,294,967,296 \\
\hline TIME & Unix Timestamp & 32 &  &  \\
\hline REAL & IEC 754 Floating Point & 32 &  &  \\
\hline LWORD & Bit String & 64 & 0 & $2^{64}$ \\
\hline LINT & Long Integer & 64 & $-2^{63}$ & $2^{63}-1$ \\
\hline ULINT & Unsigned Long Integer & 64 & 0 & $2^{64}$ \\
\hline LREAL & IEC 754 Double Floating Point & 64 &  &  \\
\hline 
\end{tabular}
 
The opendax.h header file contains the definitions for these data types.  These definitions are the name of the data type in the above table prefixed with "\texttt{DAX\_}".  So to represent a Boolean data type to the OpenDAX library you would use the definition \texttt{DAX\_BOOL}.  These definitions are used anytime your module needs to communicate a data type to the OpenDAX library.  For example, when creating a tag the module has to specify the data type.  You could create an INT tag with a call like this...

\begin{verbatim}
dax_add_tag("MyInt", DAX_INT, 1);
\end{verbatim}

This would create a tag in the server with the name \textit{MyInt} of type INT.  The 1 as the last argument just means a single member.  A larger number would signify an array.

There are also some typedefs in opendax.h that help with declaring variables within you module code.  They are the same as the precompiler defines except they are all lower case.  These make sure that the variable definitions inside your module match the data types of the tags in OpenDAX.  For the above Tag you could create a variable in C with this code...

\begin{verbatim}
dax_int myInt;
myInt = 13000;
\end{verbatim}

Each Tag in OpenDAX can be a single value of any of these base data types or it can be an array of these.  

\begin{verbatim}
dax_add_tag("MyInt", DAX_INT, 10);
\end{verbatim}

This code would generate an array of INT's in the Server.  It is important to note that tags can be redefined as long as they data type stays the same.  If you call dax\_add\_tag() again with a count that is higher than the previous call it would increase the size of the array.  If you call it with a count smaller than the previous, it will ignore it and keep the array the same size.  If you change the data type in the second call the function will return an error.

OpenDAX also supports the concept of a \textit{Compound Data Type}\index{Compound Data Type}.  This is an aggregate data type that is very similar to a structure in C.  Your module can create a new compound data type or it can use ones that are created by other modules.

These are the four functions that we will need to use from the libdax library.

\begin{verbatim}
dax_cdt *dax_cdt_new(char *name, int *error);

int dax_cdt_member(dax_state *ds, dax_cdt *cdt, char *name,
                   tag_type mem_type, unsigned int count);

int dax_cdt_create(dax_state *ds, dax_cdt *cdt, tag_type *type);

void dax_cdt_free(dax_cdt *cdt);
\end{verbatim}

To create a CDT you first have to allocate a CDT object.

The \verb|dax_cdt_new()|\index{dax\_cdt\_new() function} allocates, initializes and returns a pointer to a new CDT object.  The \textit{name} argument is the name that would be given to the CDT.  The \textit{error} argument is a pointer to an integer that can indicate any errors.  If there is an error the function will return NULL and the integer pointed to \textit{error} will be set to the error code.  If you are not interested in this error code then you can pass NULL to the function.

Once you have the object you add members to the CDT one at a time.  These members can be of any previously defined data type including other CDTs.  They can also be arrays and even arrays of other CDT's.  We do this with the \verb|dax_cdt_member()|\index{dax\_cdt\_member() function}.  The \textit{cdt} argument is the object that was returned from the \verb|dax_cdt_new()| function.  The \textit{name} argument is the name that we want to give to our member.  The \textit{mem\_type} argument is either one of the 15 base DAX\_* datatypes from above or a predefined compound data type.

Once all of the members have been added the module the \verb|dax_cdt_create()|\index{dax\_cdt\_create() function} function is used to send the data type definition to the server and actually create the data type.  The \textit{type} argument to \verb|dax_cdt_create()| will contain the new type identifier of the created compound data type.  This identifier can be used anywhere another data type definition (such as DAX\_DINT) could be used.

The \verb|dax_cdt_free()|\index{dax\_cdt\_free() function} function simply frees the memory associated with the new data type.  Don't try to free the memory yourself because there are other data structures in the \verb|dax_cdt| object that have to be freed.  Simply passing the pointer to \verb|free()| for example will result in a memory leak.  Once the data type has been created it can be freed.  Obviously failure to do this will also result in a memory leak.  Don't try to reuse a \verb|dax_cdt| object either.  You can reuse the pointer but you need to free the old one with \verb|dax_cdt_free()| and then reallocate a new one with \verb|dax_cdt_new()|.

Perhaps it's time for an example.  Let's build a CDT that has the following structure...

\begin{verbatim}
MyCDT
  MyInteger INT
  MyReals   REAL[10]
  MyBools   BOOL[8]
\end{verbatim}

The following code would be used\ldots
\begin{verbatim}
dax_cdt *dc;
int error;
tag_type type;

dc = dax_new("MyCDT", &error);
if(dc == NULL) {
    printf("Error code returned is %d\n", error);
    //handle error
} else {
    dax_cdt_member(ds, dc, "MyInteger", DAX_INT, 1);
    dax_cdt_member(ds, dc, "MyReals", DAX_REAL, 10);
    dax_cdt_member(ds, dc, "MyBools", DAX_BOOL, 8);

    error = dax_cdt_create(ds, dc, &type);
    if(error) {
        printf("Error returned from dax_cdt_member() is %d\n", error);
	    //handle error
    } else {
        printf("CDT created!  Data Type = 0x%X\n", type);
    }
    dax_cdt_free(dc);
}

\end{verbatim}

This code will build the above CDT.  There is some error checking missing on the return values of \verb|dax_cdt_member()| functions but this is for clarity.  In your code you would want to check these return values.  Also be careful of the error handling so that you don't exit this code before the \verb|dax_cdt_free()| function has been called to free the \verb|dax_cdt| object.

\section{Creating Tags}

\section{Reading Data}

\section{Writing Data}


\chapter{Handling Events}

Event handling is not part of the system yet.  This is still a work in progress.

\chapter{Messaging to Other Modules}

Module messaging is not part of the system yet.  This is still a work in progress.

\chapter{Shell Modules}
Shell Modules are called that because they are processes that are normally started from the command line shell.  These modules could be anything from an mp3 player to a database client.  They could be just about any program that can be started from the shell prompt.  They obviously don't have any "normal" OpenDAX functionality.  OpenDAX allows their use by letting other modules gain access to the shell modules STDIN, STDOUT and STDERR file descriptors.  Text can be sent from other modules to these shell modules and controlled as though that text was being typed on the command line.  This allows OpenDAX to easily add functionality found in other programs and perhaps not reinvent too many wheels.

Shell modules must be started by the \opendax server.  If you think about it this only makes sense.  If you simply ran them from the command line they would just be the same old programs.  When they are run as children of the \opendax server, the server would have access to the file descriptors that it needs to manipulate these programs.  It becomes a wrapper process for these programs.

These would be configured in \textit{opendax.conf}.

\appendix

\chapter{Library API Reference}
\section{General}
\begin{verbatim}
dax_state
\end{verbatim}
\index{dax\_state object}

\begin{verbatim}
dax_state *dax_init(char *name)
\end{verbatim}
\index{dax\_init() function}

\begin{verbatim}
int dax_free(dax_state *ds)
\end{verbatim}
\index{dax\_init() function}

\section{Configuration}

\begin{verbatim}
dax_init_config(dax_state *ds, char *name)
\end{verbatim}
\index{dax\_init\_config() function}

\begin{verbatim}
int dax_add_attribute(dax_state *ds, char *name, char *longopt,
                        char shortopt, int flags, char *defvalue)
\end{verbatim}
\index{dax\_add\_attribute() function}

\begin{verbatim}
int dax_configure(dax_state *ds, int argc, char **argv, int flags)
\end{verbatim}
\index{dax\_configure() function}

\begin{verbatim}
int dax_attr_callback(dax_state *ds, char *name,
                      int (*attr_callback)(char *name, char *value))
\end{verbatim}
\index{dax\_attr\_callback() function}

\begin{verbatim}
int dax_set_luafunction(dax_state *ds, int (*f)(void *L), char *name)
\end{verbatim}
\index{dax\_set\_luafunction() function}

\begin{verbatim}
char *dax_get_attr(dax_state *ds, char *name)
\end{verbatim}
\index{dax\_get\_attr() function}

\begin{verbatim}
int dax_set_attr(dax_state *ds, char *name, char *value)
\end{verbatim}
\index{dax\_set\_attr() function}

\begin{verbatim}
int dax_free_config(dax_state *ds)
\end{verbatim}
\index{dax\_free\_config() function}

\section{Data Handling}


\section{Event Handling}
\begin{verbatim}
int dax_event_add(dax_state *ds, Handle *handle, int event_type,
                  void *data, dax_event_id *id, 
                  void (*callback)(void *udata), void *udata);
\end{verbatim}
\index{dax\_event\_add() function}

\begin{verbatim}
int dax_event_del(dax_state *ds, dax_event_id id);
\end{verbatim}
\index{dax\_event\_del() function}

\begin{verbatim}
int dax_event_select(dax_state *ds, int timeout, dax_event_id *id);
\end{verbatim}
\index{dax\_event\_select() function}

\begin{verbatim}
int dax_event_poll(dax_state *ds, dax_event_id *id);
\end{verbatim}
\index{dax\_event\_poll() function}

\begin{verbatim}
int dax_event_get_fd(dax_state *ds);
\end{verbatim}
\index{dax\_event\_get\_fd() function}

\begin{verbatim}
int dax_event_dispatch(dax_state *ds, dax_event_id *id);
\end{verbatim}
\index{dax\_event\_dispatch() function}


%\chapter{Skeleton Module}
%\verbatiminput{../modules/skel/skel.c}

\chapter{Codebase Descriptions}
This section describes the different files and directories that are part of the OpenDAX core program (the server and the library).  This is meant as a rough overview to get a feel for how the code is laid out.  It is not meant to be an exhaustive reference.  For that you'll just have to go take a look at the latest code in the repository.  As a module developer you may not need any of this information, but since OpenDAX is such a young program there will still be plenty of bugs to work out of the core system, and this list may help somebody to get into the core code and help out with debugging or development of the server or library.

\emph{/config.h}
This file is generated by ./configure and contains the architecture dependent defines that tell us things like what functions are available and what header files we need to install. I'm actually checking for quite a few functions that I don't really do anthing about if they are missing. As incompatibilities come up these will have to be dealt with.

\emph{/common.h}

This header includes config.h as well as some other headers that are popular. It also has some definitions and macros that will be used throughout the system. This file should be included in just about every source code file in the system.

\emph{/opendax.h}

This is the header that describes the public interface to the OpenDAX library. It contains declarations for all of the public library functions as well as the precompiler definitions for the data types, error codes, configuration flags etc. This file should be included by all modules that will link to the library, and is included in most of the source code files in the rest of the system too. This should be the only header file from this distribution that would need to be included in any module code. If there are others then we did something wrong in the interface.

\emph{/dax}

The dax directory contains the source for the OpenDAX server.

\emph{/dax/daxtypes.h}

This contains the private type definitions that are used internally by the server. These definitions should not be used by any module or the library.

\emph{/dax/libcommon.h}

This file contains type definitions that are common between the library and the server. It should be included in the server and the libary source.

\emph{/dax/func.c}

Contains a few generic functions for common operations like memory allocation and such.

\emph{/dax/module.c}

Contains the code for the module handling system in the server. Operations such as starting and stopping modules as well as module registration and any other operation that involves modules should be in this file. The module.h header contains the public interface definitions for the functions in this file.

\emph{/dax/message.c}

This file contains the functions that handle the module<->server messaging. Very little actual work gets done here other than sending and receiving data on the sockets and determining which functions in other files should be called.

\emph{/dax/buffer.c}

Contains buffering code for the messaging subsystem. The messaging system writes the data from each socket into these buffers until it figures out that it has an entire message from one of the modules then it calls the function to deal with that message.

\emph{/dax/options.c}

Contains the code for reading the configuraiton from the configuration file and the command line.

\emph{/dax/opendax.c}

Contains main() and the other functions necessary to start the server and spawn the threads that do all the work. It all starts here.

\emph{/dax/tagbase.c}

The storage and maipulation of all the real time tag information is contained within this file. For now the custom datatype handling code is also in here but this may get moved to it's own file.

\emph{/lib}

The lib directory contains the source for the library. We use libtool to sort out the compatibility issues associated with the way different systems handle shared libraries. If no shared libary system will work then libtool makes this a static library.

\emph{/lib/libdax.h}

This header file contains all of the private definitions, macros and declarations that are needed throughout the library code but are not needed in the server or the modules.

\emph{/lib/libfunc.c}

Contains some generic functions that are useful throughout.

\emph{/lib/libmsg.c}

This is all the messaging code for the library. There are functions in this file for sending and retrieving messages and determining what to do with them. The functions here closely resemble functions in /dax/message.c and typically changes in one of thse files requires changes in the other. Between the two of them they define the
communications protocol.

\emph{/lib/data.c}

This file deals with library side of the tag data. It is probably not named well and may change. The functions here resemble functions that are in /dax/tagname.c

\emph{/lib/libconv.c}

This file contains the functions for making sure that the data formatting is the same as the server. The way that OpenDAX handles different byte ordering and data formating between architectures over the network is that the server stores the data in whatever way the server wants and the library is responsible for determining if the data needs to be converted and how. This file contains the code for that.

\emph{/lib/libcdt.c}

OpenDAX allows the creation of custom data types. This file contains the code to handle all of that.

\emph{/lib/libopt.c}

Contains the functions for configuring the module. The entire configuration system of OpenDAX uses Lua as the configuration programming language. The modules can be configured by either the main opendax.conf file, their own configuration file, the command line or any combination of the above. There are also some configuration options that are common between modules. This file contains the code to handle all of that.

\emph{/modules}

This directory contains the module code. We will not discuss the module details here.

\emph{/etc}

The sample configuration files are located here.

\chapter{opendax.h Listing}
\verbatiminput{../opendax.h}

%\chapter{GPL License}
%\verbatiminput{../COPYING}

\printindex
\end{document}
